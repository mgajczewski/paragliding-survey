\documentclass[11pt,a4paper]{article}

% Packages
\usepackage[utf8]{inputenc}
\usepackage[T1]{fontenc}
\usepackage{geometry}
\usepackage{enumitem}
\usepackage{amssymb}
\usepackage{textcomp}
\usepackage{hyperref}
\usepackage{parskip}

% Page layout
\geometry{margin=1in}

% Custom commands
\newcommand{\checkbox}{\makebox[0pt][l]{$\square$}\raisebox{.15ex}{\hspace{0.1em}$\square$}}

\title{\textbf{Survey: Ballast \& Fairness in Paragliding Competition}}
\author{}
\date{}

\begin{document}

\maketitle

\section*{Introduction}

Thank you for participating in this crucial survey.

We are gathering data from pilots and designers to find a widely supported solution to the issues of excessive ballast, safety, and fairness in paragliding competitions.

Your anonymous responses will help guide future rule-making and ensure the sport remains safe, fair, and inclusive.

This survey should take approximately \textbf{10--15 minutes}.

\section{About You (Demographics)}

This information is essential for analyzing how different groups perceive the issues.

\textbf{1. What is your primary role in paragliding?}
\begin{itemize}[label=$\square$]
    \item Competition Pilot (Cat 1 / PWC)
    \item Competition Pilot (Cat 2 / National League)
    \item Recreational / XC Pilot (non-competition)
    \item Paraglider Designer / Test Pilot
    \item Manufacturer Representative
    \item Competition Organizer
    \item Other: \underline{\hspace{3cm}}
\end{itemize}

\textbf{2. What is your naked body weight?}
\begin{itemize}[label=$\square$]
    \item $<$ 50 kg (110 lbs)
    \item 50--59 kg (110--130 lbs)
    \item 60--69 kg (131--152 lbs)
    \item 70--79 kg (153--174 lbs)
    \item 80--89 kg (175--196 lbs)
    \item 90--99 kg (197--218 lbs)
    \item 100+ kg (219+ lbs)
\end{itemize}

\textbf{3. What is your typical competition all-up weight (AUW) in flight?}
\begin{itemize}[label=$\square$]
    \item $<$ 75 kg
    \item 75--84 kg
    \item 85--94 kg
    \item 95--104 kg
    \item 105--114 kg
    \item 115--124 kg
    \item 125+ kg
\end{itemize}

\textbf{4. How much ballast (lead, water, etc.) do you typically carry in a competition?}
\begin{itemize}[label=$\square$]
    \item 0 kg (ballast-free)
    \item 1--5 kg
    \item 6--10 kg
    \item 11--15 kg
    \item 16--20 kg
    \item 21+ kg
\end{itemize}

\textbf{5. What is the size (e.g., M) and category (e.g., CCC) of your primary competition wing?}
\begin{itemize}[label={}]
    \item Size: \underline{\hspace{3cm}}
    \item Category: \underline{\hspace{3cm}}
\end{itemize}

\section{Problem Perception}

How do you see the current situation?

\textbf{6. On a scale of 1 to 5, how significant is the safety risk associated with pilots carrying large amounts of ballast?}
\begin{itemize}[label={}]
    \item (1) Not significant
    \item (2) Slightly significant
    \item (3) Moderately significant
    \item (4) Very significant
    \item (5) Extremely significant (a critical issue)
\end{itemize}

\textbf{7. On a scale of 1 to 5, how significant is the competitive fairness issue related to pilot weight and ballasting?}
\begin{itemize}[label={}]
    \item (1) Not an issue
    \item (2) Slight issue
    \item (3) Moderate issue
    \item (4) Significant issue
    \item (5) Extremely unfair (a critical issue)
\end{itemize}

\textbf{8. Which statement best reflects your personal philosophy?}
\begin{itemize}[label=$\square$]
    \item Paragliding competition should be about pilot skill and decision-making; rules should actively equalize advantages/disadvantages from body weight.
    \item Physical attributes (like weight) are a natural part of any sport; rules should not try to compensate for these differences.
    \item Rules should only intervene to address extreme safety risks (like $>$20kg ballast), but should not try to engineer competitive fairness.
\end{itemize}

\section{Evaluating the Solutions}

We will now ask your opinion on the main solutions proposed.

\subsection{Solution 1: ``Equalizers'' (Noodles)}

\textit{(As proposed by Luc Armant -- adding drag elements to larger wings to equalize glide performance across sizes)}

\textbf{9. How familiar are you with the ``Equalizers'' proposal?}
\begin{itemize}[label=$\square$]
    \item Not familiar
    \item Vaguely familiar (heard the name)
    \item Familiar (understand the concept)
    \item Very familiar (read the Gaggler report document)
\end{itemize}

\textbf{10. Please rate your overall support for testing/implementing ``Equalizers'':}
\begin{itemize}[label=$\square$]
    \item Strongly Oppose
    \item Oppose
    \item Neutral / Need more info
    \item Support
    \item Strongly Support
\end{itemize}

\textbf{11. What are your biggest concerns about ``Equalizers''? (Check all that apply)}
\begin{itemize}[label=$\square$]
    \item Safety/Certification: Unsure how they affect handling, collapses, or SIV maneuvers.
    \item Installation: Pilots might install them incorrectly, creating a safety risk.
    \item Unfairness: Unfairly ``punishes'' heavier pilots who have no choice but to fly large wings.
    \item Ineffective: Won't truly equalize performance (e.g., handling, climb).
    \item Ineffective: Won't stop light pilots from ballasting up to M-sizes for better ``comfort'' or ``handling.''
    \item Market: Won't actually incentivize manufacturers to build better small wings.
    \item Philosophy: It feels ``absurd'' to intentionally add drag in a performance sport.
    \item Other: \underline{\hspace{3cm}}
\end{itemize}

\textbf{12. What do you see as the biggest benefits of ``Equalizers''? (Check all that apply)}
\begin{itemize}[label=$\square$]
    \item Levels the field: Makes performance more equal across wing sizes.
    \item Reduces ballast: Removes the incentive for S/M pilots to ballast up to M/L sizes.
    \item Promotes small wings: Creates a real incentive for manufacturers to develop better XS/S wings.
    \item Keeps ``Overall'' ranking: Allows everyone to race together and preserves a single ``Overall'' winner.
    \item Targeted solution: Directly addresses the performance gap without complex rule changes.
    \item Other: \underline{\hspace{3cm}}
\end{itemize}

\subsection{Solution 2: Weight Classes}

\textit{(Creating 4--6 distinct weight classes, e.g., ``70--80kg,'' ``80--90kg.'' This would likely replace a single ``Overall'' winner with winners for each class.)}

\textbf{13. Please rate your overall support for implementing ``Weight Classes'':}
\begin{itemize}[label=$\square$]
    \item Strongly Oppose
    \item Oppose
    \item Neutral / Need more info
    \item Support
    \item Strongly Support
\end{itemize}

\textbf{14. What are your biggest concerns about ``Weight Classes''? (Check all that apply)}
\begin{itemize}[label=$\square$]
    \item Loses ``Overall'' Winner: Destroys the prestige and simplicity of having one overall champion.
    \item Logistics: Too complicated to manage (e.g., separate starts, scoring, prizes).
    \item ``Ballasting to the limit'': Pilots will just carry ballast to be at the top of their 10kg-wide class.
    \item Small Categories: Some classes might not have enough pilots to be competitive.
    \item Enforcement: Requires strict weigh-ins, which are hard to manage and can be cheated (e.g., water loading).
    \item Team/Nation: Complicates FAI-1 team selection and scoring.
    \item Other: \underline{\hspace{3cm}}
\end{itemize}

\textbf{15. What do you see as the biggest benefits of ``Weight Classes''? (Check all that apply)}
\begin{itemize}[label=$\square$]
    \item Ultimate Fairness: Pilots compete only against those of similar weight.
    \item Eliminates Ballast: Creates a strong incentive to fly at one's ``natural'' weight.
    \item Promotes Small Wings: Creates a protected, competitive market for XS/S wings.
    \item Improves Safety: Massively reduces/eliminates the need for dangerous ballast loads.
    \item Proven Concept: Works well in many other sports (e.g., boxing, sailing).
    \item Other: \underline{\hspace{3cm}}
\end{itemize}

\subsection{Solution 3: Strict Ballast Limitations}

\textit{(A simple rule, e.g., ``Max 10kg of ballast allowed'' or ``Ballast cannot exceed 15\% of body weight.'')}

\textbf{16. Please rate your overall support for implementing ``Strict Ballast Limitations'':}
\begin{itemize}[label=$\square$]
    \item Strongly Oppose
    \item Oppose
    \item Neutral / Need more info
    \item Support
    \item Strongly Support
\end{itemize}

\textbf{17. What are your biggest concerns about ``Ballast Limitations''? (Check all that apply)}
\begin{itemize}[label=$\square$]
    \item Enforcement: Extremely difficult to enforce; pilots will hide ballast (lead in shoes, drinking water, heavy gear).
    \item Unfair: Penalizes light pilots who need some ballast just to fly an S-size wing safely, while not affecting heavy pilots.
    \item Doesn't solve it: Doesn't fix the reason pilots ballast (wing performance disparity). L-size pilots will still have an advantage.
    \item Complicated: Creates ``grey areas'' (what counts as ballast vs. gear?).
    \item Other: \underline{\hspace{3cm}}
\end{itemize}

\textbf{18. What do you see as the biggest benefits of ``Ballast Limitations''? (Check all that apply)}
\begin{itemize}[label=$\square$]
    \item Simple Rule: Easy to understand (if not to enforce).
    \item Directly addresses safety: Puts a hard stop on the most dangerous ballast loads.
    \item Partial Incentive: May encourage some pilots to move to a smaller wing.
    \item Other: \underline{\hspace{3cm}}
\end{itemize}

\subsection{Solution 4: MRT (Minimum Race Time) Scoring}

\textit{(A scoring adjustment based on wing size/AUW, as described on Gaggler.org.)}

\textbf{19. How familiar are you with the ``MRT'' proposal?}
\begin{itemize}[label=$\square$]
    \item Not familiar
    \item Vaguely familiar (heard the name)
    \item Familiar (understand the concept)
    \item Very familiar (read the Gaggler report document)
\end{itemize}

\textbf{20. Please rate your overall support for testing/implementing ``MRT'':}
\begin{itemize}[label=$\square$]
    \item Strongly Oppose
    \item Oppose
    \item Neutral / Need more info
    \item Support
    \item Strongly Support
\end{itemize}

\textbf{21. What are your biggest concerns about ``MRT''? (Check all that apply)}
\begin{itemize}[label=$\square$]
    \item Randomness: Interferes with gaggle flying; scoring becomes ``random'' depending on group composition.
    \item Safety: May create ``unexpected turns in a large group.''
    \item Too Complex: The formula is opaque, and pilots won't understand how they are being scored.
    \item Wrong Focus: Moves the sport away from racing ``against each other'' (which pilots enjoy).
    \item Ineffective: A scoring ``fix'' doesn't solve the physical problem of pilots carrying dangerous ballast.
    \item Other: \underline{\hspace{3cm}}
\end{itemize}

\textbf{22. What do you see as the biggest benefits of ``MRT''? (Check all that apply)}
\begin{itemize}[label=$\square$]
    \item Removes Ballast Incentive: Directly compensates for wing size performance, making ballast unnecessary.
    \item Keeps ``Overall'' ranking: Allows everyone to race together and preserves a single ``Overall'' winner.
    \item No physical modification: Doesn't require adding ``noodles'' or new gear.
    \item Other: \underline{\hspace{3cm}}
\end{itemize}

\section{Comparative Ranking \& Final Priorities}

\textbf{23. Please rank your preferred solutions, from 1 (Most Preferred) to 6 (Least Preferred):}
\begin{itemize}[label={}]
    \item \underline{\hspace{0.5cm}} Equalizers (``Noodles'')
    \item \underline{\hspace{0.5cm}} Weight Classes (replaces ``Overall'' winner)
    \item \underline{\hspace{0.5cm}} Strict Ballast Limitations (e.g., max 10kg)
    \item \underline{\hspace{0.5cm}} MRT (Minimum Race Time scoring)
    \item \underline{\hspace{0.5cm}} Separate ``Lightweight'' Class (e.g., $<$95kg, like the ``Reynolds'' class, but ``Overall'' winner still exists)
    \item \underline{\hspace{0.5cm}} No Change (Status Quo)
\end{itemize}

\textbf{24. A separate, but related, safety issue discussed is the design of modern harnesses (e.g., ``submarine'' types). How do you prioritize this?}
\begin{itemize}[label=$\square$]
    \item The harness safety issue is more urgent than the ballast issue and should be fixed first.
    \item The harness safety issue is equally urgent as the ballast issue.
    \item The harness safety issue is less urgent than the ballast issue.
    \item The two issues are separate and should be addressed independently.
\end{itemize}

\textbf{25. Do you have any final comments or suggestions regarding the ballast issue or these proposed solutions?}

\vspace{3cm}
\hrule

\section*{End of Survey}

Thank you for your valuable input.

\end{document}


